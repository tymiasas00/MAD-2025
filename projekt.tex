\documentclass[a4paper,12pt]{article}
\usepackage[utf8]{inputenc}
\usepackage[polish]{babel}
\usepackage[T1]{fontenc}
\usepackage{geometry}
\usepackage{graphicx}
\usepackage{amsmath}
\usepackage{booktabs}
\usepackage{hyperref}
\usepackage{float}
\geometry{margin=2.5cm}

\title{Analiza trendów i klasyfikacja filmów\\na podstawie recenzji z Rotten Tomatoes\\\large Projekt nr 1 z przedmiotu Metody Analizy Danych}
\author{Julia Łyszkowska, nr indeksu\\Tymoteusz Majewski , nr indeksu\\Szymon Marciniak, s223526\\Natan Misztal, s223309\\Wiktor Koprowski, s223372}
\date{\today}

\begin{document}

\maketitle

\begin{abstract}
% Tu należy wpisać krótkie streszczenie projektu (maks. 150 słów)
\end{abstract}

\noindent \textbf{Słowa kluczowe:} analiza skupień, klasyfikacja, filmy, recenzje, Rotten Tomatoes, eksploracja danych

\section{Wprowadzenie}
% Krótkie wprowadzenie do problematyki – znaczenie analizy danych w kontekście opinii i trendów filmowych, dlaczego Rotten Tomatoes, krótki opis używanej metody (analiza skupień)

\section{Przedmiot badania}

\subsection{Cel i zakres badania}
% Co badacie, jaki jest główny cel, jakiego typu dane i jakie aspekty filmów analizujecie

\subsection{Przegląd literatury}
% Co najmniej jedna pozycja literatury, która porusza podobną tematykę (np. klasyfikacja filmów, analiza recenzji, użycie analizy skupień itp.) + krótki opis

\subsection{Opis danych}
% Link do źródła: \url{https://www.kaggle.com/datasets/andrezaza/clapper-massive-rotten-tomatoes-movies-and-reviews}
% Krótkie przedstawienie co zawiera zbiór, ile ma rekordów, jakie typy danych (np. oceny, opisy, tagi)

\subsection{Zmienne wybrane do analizy}
% Opis i uzasadnienie wyboru co najmniej sześciu zmiennych (np. ocena krytyków, ocena użytkowników, liczba recenzji itp.)
% Podział na stymulanty/destymulanty

\subsection{Wstępna analiza danych}

\subsubsection{Statystyki opisowe}
% Tabela z podstawowymi statystykami: średnia, mediana, min, max, odchylenie standardowe, skośność

\subsubsection{Wizualizacja danych}
% Wstawienie histogramów, boxplotów itp.

\subsubsection{Braki danych}
% Jakie dane są brakujące i jak je uzupełniono/usunięto

\subsubsection{Obserwacje odstające}
% Czy występują, jak zostały obsłużone

\section{Opis zastosowanych metod}

\subsection{Analiza skupień}
% Opis metody k-średnich, metody Warda itp.
% Wzory (np. na odległość Euklidesową, SSE, silhouette score) z opisami symboli
% Cytowanie źródeł, gdzie metody zostały użyte/zaproponowane

\section{Rezultaty}

\subsection{Wyniki analizy}
% Prezentacja wyników w tabelach i wykresach (np. PCA, dendrogramy, porównanie klastrów)

\subsection{Porównanie metod}
% Porównanie wyników uzyskanych za pomocą różnych metod grupowania (np. silhouette score)

\section{Podsumowanie}
% Ocena czy cel został zrealizowany
% Odniesienie do literatury – czy wyniki są zgodne z badaniami
% Możliwe kierunki rozwoju analizy, ograniczenia

\section{Bibliografia}
% \begin{thebibliography}{9}
% \bibitem{autor2020} Nazwisko I., Tytuł pracy, Czasopismo/Raport, Rok.
% \end{thebibliography}

\end{document}
