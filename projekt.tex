\documentclass[a4paper,12pt]{article}
\usepackage[utf8]{inputenc}
\usepackage[polish]{babel}
\usepackage[T1]{fontenc}
\usepackage{geometry}
\usepackage{graphicx}
\usepackage{amsmath}
\usepackage{booktabs}
\usepackage{hyperref}
\usepackage{float}
\geometry{margin=2.5cm}

\title{Analiza trendów i klasyfikacja filmów\\na podstawie recenzji z Rotten Tomatoes\\\large Projekt nr 1 z przedmiotu Metody Analizy Danych}
\author{Julia Łyszkowska, nr indeksu\\Tymoteusz Majewski , nr indeksu\\Szymon Marciniak, s223526\\Natan Misztal, s223309\\Wiktor Koprowski, s223372}
\date{\today}

\begin{document}

\maketitle

\begin{abstract}
% Tu należy wpisać krótkie streszczenie projektu (maks. 150 słów)
\end{abstract}

\noindent \textbf{Słowa kluczowe:} analiza skupień, klasyfikacja, filmy, recenzje, Rotten Tomatoes, eksploracja danych

\section{Wprowadzenie}

W dobie rosnącej popularności platform streamingowych i serwisów z recenzjami, takich jak Rotten Tomatoes, analiza opinii o filmach stała się istotnym narzędziem w ocenie trendów kulturowych i preferencji odbiorców. Serwis Rotten Tomatoes gromadzi zarówno profesjonalne recenzje krytyków filmowych, jak i opinie zwykłych widzów, oferując cenny zbiór danych do analiz porównawczych. 

Celem niniejszego projektu jest przeprowadzenie analizy skupień na zbiorze danych filmowych z Rotten Tomatoes, w celu wyodrębnienia grup filmów o podobnych cechach i ocenach. W szczególności interesuje nas, czy możliwe jest zaobserwowanie wyraźnych zależności między gatunkiem filmu, jego oceną, językiem, długością trwania, czy udziałem znanych reżyserów i topowych krytyków. Dodatkowym aspektem badania może być również analiza zmian w ocenach filmów na przestrzeni lat, jeśli dane na to pozwolą.

Analiza obejmie zarówno oceny krytyków, jak i widzów, a także syntetyczne miary sentymentu recenzji tekstowych (pozytywna/negatywna). Zakładamy, że dzięki zastosowaniu metod eksploracyjnych, takich jak analiza skupień, możliwe będzie znalezienie interesujących grup filmów – np. tych, które cieszą się wysoką popularnością wśród widzów, ale są nisko oceniane przez krytyków. Jedną z hipotez, którą chcemy zweryfikować, jest założenie, że filmy z popularnych, „prostych” gatunków (np. komedie romantyczne, filmy akcji) uzyskują wyższe oceny od widzów niż od profesjonalnych recenzentów.

Zbiór danych użyty w analizie pochodzi z publicznie dostępnego repozytorium na platformie Kaggle i został częściowo wstępnie przetworzony, co umożliwia przeprowadzenie analizy skupień z użyciem narzędzi języka Python.


\section{Przedmiot badania}

\subsection{Cel i zakres badania}

Celem niniejszego projektu jest analiza trendów oraz klasyfikacja filmów na podstawie recenzji z serwisu Rotten Tomatoes, z wykorzystaniem metod analizy skupień. Badanie obejmuje zarówno oceny krytyków, jak i widzów, a także inne atrybuty filmów, takie jak gatunek, długość trwania, reżyser, oryginalny język oraz sentyment recenzji tekstowych. Analiza skupień pozwoli na identyfikację grup filmów o podobnych cechach i ocenach, co może ujawnić ukryte zależności i wzorce w danych.

\subsection{Przegląd literatury}

Analiza recenzji filmowych z wykorzystaniem metod eksploracji danych jest przedmiotem licznych badań. Xu i in. \cite{xu2022} zastosowali techniki analizy skupień do wizualizacji danych z recenzji filmowych, co pozwoliło na lepsze zrozumienie opinii widzów. Verma i in. \cite{verma2021} wykorzystali głębokie sieci neuronowe do analizy sentymentu recenzji filmowych, osiągając wysoką dokładność w przewidywaniu ocen. Abimanyu i in. \cite{abimanyu2023} przeprowadzili analizę sentymentu recenzji z Rotten Tomatoes, stosując regresję logistyczną i selekcję cech metodą Information Gain, co pozwoliło na skuteczną klasyfikację opinii. Nugraha i in. \cite{nugraha2023} zastosowali zmodyfikowaną metodę zrównoważonego lasu losowego oraz Word2Vec do analizy sentymentu recenzji filmowych, uzyskując wysoką skuteczność klasyfikacji. Powyższe badania wskazują na efektywność metod analizy skupień i analizy sentymentu w eksploracji danych z recenzji filmowych.

\subsection{Opis danych}

Dane wykorzystane w projekcie pochodzą z publicznie dostępnego zbioru na platformie Kaggle \cite{kaggle2025}, zawierającego obszerne informacje o filmach i ich recenzjach z serwisu Rotten Tomatoes. Zbiór obejmuje zarówno oceny krytyków, jak i widzów, a także metadane filmów, takie jak tytuł, gatunek, długość trwania, reżyser, oryginalny język oraz teksty recenzji. Dane te zostały wstępnie przetworzone w celu usunięcia brakujących wartości i nieistotnych kolumn, co umożliwia efektywną analizę.

\subsection{Zmienne wybrane do analizy}

Do analizy wybrano następujące zmienne:

\begin{itemize}
    \item \textbf{Ocena krytyków (Critics' Score)}: średnia ocena filmu przyznana przez krytyków.
    \item \textbf{Ocena widzów (Audience Score)}: średnia ocena filmu przyznana przez widzów.
    \item \textbf{Gatunek (Genre)}: kategoria, do której przypisany jest film (np. dramat, komedia, akcja).
    \item \textbf{Długość trwania (Runtime)}: czas trwania filmu w minutach.
    \item \textbf{Reżyser (Director)}: nazwisko reżysera filmu.
    \item \textbf{Oryginalny język (Original Language)}: język, w którym film został pierwotnie wyprodukowany.
    \item \textbf{Sentyment recenzji (Review Sentiment)}: klasyfikacja recenzji tekstowych jako pozytywne lub negatywne.
    \item \textbf{Status krytyka (Top Critic Status)}: informacja, czy krytyk jest uznawany za czołowego krytyka.
\end{itemize}

Zmienne \textbf{Ocena krytyków}, \textbf{Ocena widzów}, \textbf{Sentyment recenzji} oraz \textbf{Status krytyka} traktowane są jako stymulanty, ponieważ wyższe wartości wskazują na lepsze oceny. Pozostałe zmienne pełnią rolę atrybutów opisowych, które mogą wpływać na oceny filmów.

\subsection{Wstępna analiza danych}

\subsubsection{Statystyki opisowe}
% Tabela z podstawowymi statystykami: średnia, mediana, min, max, odchylenie standardowe, skośność
Wstępna analiza statystyczna obejmuje obliczenie podstawowych miar dla wybranych zmiennych, takich jak średnia, mediana, minimum, maksimum, odchylenie standardowe oraz skośność. Pozwoli to na zrozumienie rozkładu danych i identyfikację ewentualnych anomalii.


\subsubsection{Wizualizacja danych}
% Wstawienie histogramów, boxplotów itp.
Do wizualizacji danych wykorzystane zostaną histogramy, wykresy pudełkowe (boxplot) oraz wykresy rozrzutu, co umożliwi graficzne przedstawienie rozkładu zmiennych oraz zależności między nimi.


\subsubsection{Braki danych}
% Jakie dane są brakujące i jak je uzupełniono/usunięto
Analiza brakujących danych pozwoli na zidentyfikowanie ewentualnych niekompletnych rekordów. W zależności od skali braków, zastosowane zostaną odpowiednie metody ich uzupełnienia lub usunięcia.


\subsubsection{Obserwacje odstające}
% Czy występują, jak zostały obsłużone
Identyfikacja obserwacji odstających zostanie przeprowadzona za pomocą metod statystycznych, takich jak analiza rozkładu oraz wykresy pudełkowe. Wykryte wartości odstające zostaną poddane analizie w celu podjęcia decyzji o ich ewentualnym usunięciu lub pozostawieniu w zbiorze danych.


\section{Opis zastosowanych metod}

\subsection{Analiza skupień}
% Opis metody k-średnich, metody Warda itp.
% Wzory (np. na odległość Euklidesową, SSE, silhouette score) z opisami symboli
% Cytowanie źródeł, gdzie metody zostały użyte/zaproponowane

\section{Rezultaty}

\subsection{Wyniki analizy}
% Prezentacja wyników w tabelach i wykresach (np. PCA, dendrogramy, porównanie klastrów)

\subsection{Porównanie metod}
% Porównanie wyników uzyskanych za pomocą różnych metod grupowania (np. silhouette score)

\section{Podsumowanie}
% Ocena czy cel został zrealizowany
% Odniesienie do literatury – czy wyniki są zgodne z badaniami
% Możliwe kierunki rozwoju analizy, ograniczenia

\section{Bibliografia}
\begin{thebibliography}{9}

\bibitem{xu2022} Xu, Y., et al. (2022). Application of Cluster Analysis Technology in Visualization Research of Movie Review Data. \textit{Computational Intelligence and Neuroscience}. \url{https://doi.org/10.1155/2022/7756896}.

\bibitem{verma2021} Verma, V., Gupta, A., Kumawat, S. (2021). Sentiment Analysis for Movie Ratings Using Deep Learning. In: \textit{Data Engineering and Intelligent Computing}. Springer, Singapore. \url{https://doi.org/10.1007/978-981-16-0171-2_6}.

\bibitem{abimanyu2023} Abimanyu, A. J., Dwifebri, M., Astuti, H. (2023). Sentiment Analysis of Rotten Tomatoes Movie Reviews Using Logistic Regression and Information Gain Feature Selection. \textit{Journal of Physics: Conference Series}, 2630(1). \url{https://doi.org/10.1088/1742-6596/2630/1/012004}.

\bibitem{nugraha2023} Nugraha, A. T., Yuwono, H. (2023). Sentiment Analysis of Rotten Tomatoes Movie Reviews Using Modified Balanced Random Forest and Word2Vec. \textit{TELKOMNIKA Telecommunication, Computing, Electronics and Control}, 21(1). \url{https://doi.org/10.12928/telkomnika.v21i1.23642}.

\bibitem{kaggle2025} Andrezaza. (2023). Clapper: Massive Rotten Tomatoes Movies and Reviews. \textit{Kaggle}. \url{https://www.kaggle.com/datasets/andrezaza/clapper-massive-rotten-tomatoes-movies-and-reviews}

\end{thebibliography}

\end{document}
