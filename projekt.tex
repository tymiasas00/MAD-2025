\documentclass[a4paper,12pt,titlepage]{article}
\usepackage[utf8]{inputenc}
\usepackage[polish]{babel}
\usepackage[T1]{fontenc}
\usepackage{geometry}
\usepackage{graphicx}
\usepackage{amsmath}
\usepackage{booktabs}
\usepackage{hyperref}
\usepackage{float}
\geometry{margin=2.5cm}

\title{Analiza trendów i klasyfikacja filmów\\na podstawie recenzji z Rotten Tomatoes\\\large Projekt nr 1 z przedmiotu Metody Analizy Danych}
\author{Julia Łyszkowska, s223487\\Tymoteusz Majewski , s223285\\Szymon Marciniak, s223526\\Natan Misztal, s223309\\Wiktor Koprowski, s223372}
\date{\today}

\begin{document}

\begin{titlepage}
  \maketitle
\end{titlepage}
\clearpage

\begin{abstract}
Projekt obejmował analizę trendów w ocenach filmowych na podstawie danych z serwisu Rotten Tomatoes. Głównym celem badania było zastosowanie metod analizy skupień (K-średnich, DBSCAN i klasteryzacji hierarchicznej) do identyfikacji grup filmów o podobnych charakterystykach. Analizowano zależności między ocenami krytyków (\texttt{tomatoMeter}) i widzów (\texttt{audienceScore}), czasem trwania filmów (\texttt{runtimeMinutes}), udziałem recenzji od uznanych krytyków (\texttt{topCriticRatio}) oraz sentymentem recenzji (\texttt{scoreSentiment}). Wyniki wykazały istotne różnice w ocenach między krytykami a publicznością, szczególnie dla niektórych gatunków filmowych. Metoda K-średnich z 4 klastrami oraz klasteryzacja hierarchiczna dały najbardziej czytelne podziały, podczas gdy DBSCAN skutecznie identyfikował obserwacje odstające. Projekt potwierdził przydatność metod eksploracyjnych w analizie preferencji filmowych i wykazał wartość kombinacji różnych technik klasteryzacji.\\

\noindent\textbf{Słowa kluczowe:} analiza skupień, klasyfikacja filmów, Rotten Tomatoes, eksploracja danych, K-means, DBSCAN, klasteryzacja hierarchiczna, analiza sentymentu, oceny filmowe, machine learning
\end{abstract}

\clearpage
\tableofcontents
\clearpage

\section{Wprowadzenie}

W dobie rosnącej popularności platform streamingowych i serwisów z recenzjami, takich jak Rotten Tomatoes, analiza opinii o filmach stała się istotnym narzędziem w ocenie trendów kulturowych i preferencji odbiorców. Serwis Rotten Tomatoes gromadzi zarówno profesjonalne recenzje krytyków filmowych, jak i opinie zwykłych widzów, oferując cenny zbiór danych do analiz porównawczych. 

Celem niniejszego projektu jest przeprowadzenie analizy skupień na zbiorze danych filmowych z Rotten Tomatoes, w celu wyodrębnienia grup filmów o podobnych cechach i ocenach. W szczególności interesuje nas, czy możliwe jest zaobserwowanie wyraźnych zależności między gatunkiem filmu, jego oceną, językiem, długością trwania, czy udziałem znanych reżyserów i topowych krytyków. Dodatkowym aspektem badania może być również analiza zmian w ocenach filmów na przestrzeni lat, jeśli dane na to pozwolą.

Analiza obejmie zarówno oceny krytyków, jak i widzów, a także syntetyczne miary sentymentu recenzji tekstowych (pozytywna/negatywna). Zakładamy, że dzięki zastosowaniu metod eksploracyjnych, takich jak analiza skupień, możliwe będzie znalezienie interesujących grup filmów – np. tych, które cieszą się wysoką popularnością wśród widzów, ale są nisko oceniane przez krytyków. Jedną z hipotez, którą chcemy zweryfikować, jest założenie, że filmy z popularnych, „prostych” gatunków (np. komedie romantyczne, filmy akcji) uzyskują wyższe oceny od widzów niż od profesjonalnych recenzentów.

Zbiór danych użyty w analizie pochodzi z publicznie dostępnego repozytorium na platformie Kaggle i został wstępnie przetworzony, co umożliwia przeprowadzenie analizy skupień z użyciem narzędzi języka Python.


\section{Przedmiot badania}

\subsection{Cel i zakres badania}

Celem niniejszego projektu jest analiza trendów oraz klasyfikacja filmów na podstawie recenzji z serwisu Rotten Tomatoes, z wykorzystaniem metod analizy skupień. Badanie obejmuje zarówno oceny krytyków, jak i widzów, a także inne atrybuty filmów, takie jak gatunek, długość trwania, reżyser, oryginalny język oraz sentyment recenzji tekstowych. Analiza skupień pozwoli na identyfikację grup filmów o podobnych cechach i ocenach, co może ujawnić ukryte zależności i wzorce w danych.

\subsection{Przegląd literatury}

Analiza danych filmowych z wykorzystaniem metod eksploracji danych stanowi istotny obszar współczesnych badań. W pracy Xu i in. \cite{xu2022} zastosowano zaawansowane techniki wizualizacji danych z recenzji filmowych, demonstrując ich przydatność w identyfikacji wzorców w ocenach użytkowników serwisów takich jak Rotten Tomatoes. Autorzy wykazali, że odpowiednie przetworzenie danych o filmach pozwala na wydzielenie grup o podobnych charakterystykach, co może stanowić wartość dla analizy preferencji widzów.

Badania Abimanyu i in. \cite{abimanyu2023} dostarczają istotnych wniosków dotyczących przetwarzania recenzji z platform filmowych. Autorzy skupili się na analizie sentymentu tekstów z Rotten Tomatoes, osiągając wysoką skuteczność w klasyfikacji tonacji recenzji dzięki połączeniu regresji logistycznej z selekcją cech. Ich podejście może być szczególnie wartościowe przy wstępnym przetwarzaniu danych tekstowych.

Uzupełnieniem jest praca Goutam i in. \cite{goutam2018}, którzy przeprowadzili kompleksową analizę porównawczą ocen krytyków i widzów na podstawie danych z Rotten Tomatoes. Autorzy zidentyfikowali istotne różnice w ocenach między tymi grupami, szczególnie dla określonych gatunków filmowych, co znajduje odzwierciedlenie w wynikach przedstawionych w niniejszej pracy.

Przegląd wskazuje, że analiza danych z platform agregujących recenzje filmowe stanowi wartościowe źródło informacji o preferencjach widowni i krytyków, co potwierdzają wyniki przedstawione w dalszej części pracy.

\subsection{Opis danych}

Dane wykorzystane w projekcie pochodzą z publicznie dostępnego zbioru na platformie Kaggle \cite{kaggle2025}, zawierającego obszerne informacje o filmach i ich recenzjach z serwisu Rotten Tomatoes. Zbiór obejmuje zarówno oceny krytyków, jak i widzów, a także metadane filmów, takie jak tytuł, gatunek, długość trwania, reżyser, oryginalny język oraz teksty recenzji. Dane te zostały wstępnie przetworzone w celu usunięcia brakujących wartości i nieistotnych kolumn, co umożliwia efektywną analizę.

\subsection{Zmienne wybrane do analizy}

Do dalszej analizy wybrano sześć zmiennych, które mają istotne znaczenie dla oceny filmów oraz odzwierciedlają różne aspekty zarówno opinii krytyków i widzów, jak i cechy samego filmu. Zmiennymi tymi są: \texttt{audienceScore}, \texttt{tomatoMeter}, \texttt{originalScore}, \texttt{runtimeMinutes}, \texttt{topCriticRatio} oraz \texttt{scoreSentiment}. Poniżej przedstawiono krótki opis każdej z nich:

\begin{itemize}
    \item \textbf{\texttt{audienceScore}} – zmienna numeryczna reprezentująca procent pozytywnych ocen wystawionych przez widzów. Przyjmuje wartości z przedziału od 0 do 100 i jest stymulantą, ponieważ wyższy wynik oznacza lepszy odbiór filmu przez publiczność.

    \item \textbf{\texttt{tomatoMeter}} – zmienna numeryczna odzwierciedlająca ocenę filmu przez krytyków w postaci procentu pozytywnych recenzji. Podobnie jak \texttt{audienceScore}, mieści się w zakresie od 0 do 100 i również stanowi stymulantę.

    \item \textbf{\texttt{originalScore}} – przekształcona zmienna liczbowo reprezentująca ocenę przypisaną przez pojedynczego recenzenta. Wartości tej zmiennej różnią się w zależności od formatu ocen stosowanego na stronie Rotten Tomatoes i zostały przekształcone do ujednoliconej skali. Dla danego filmu obliczono różne statystyki, m.in. wartość średnią, minimalną, maksymalną i medianę. Jako zmienna opisująca pozytywny lub negatywny odbiór filmu, jest stymulantą.

    \item \textbf{\texttt{runtimeMinutes}} – liczba minut trwania filmu. Jest to zmienna ilościowa przyjmująca wartości dodatnie, typowo w zakresie od około 60 do ponad 180 minut. Nie jest interpretowana jako stymulanta, ponieważ dłuższy czas trwania nie musi oznaczać lepszej jakości filmu.

    \item \textbf{\texttt{topCriticRatio}} – udział recenzji pochodzących od tzw. Top Critics, czyli recenzentów uznanych za najbardziej wiarygodnych i wpływowych. Jest to wartość z przedziału od 0 do 1, gdzie 1 oznacza, że wszystkie recenzje filmu pochodzą od Top Critics. Zmienna ta została uznana za stymulantę, ponieważ większy udział uznanych recenzentów może wskazywać na większą wartość filmu w oczach środowiska krytyków.

    \item \textbf{\texttt{scoreSentiment}} – binarna zmienna przyjmująca wartości 0 (recenzja negatywna) lub 1 (recenzja pozytywna). Dla każdego filmu obliczono średni sentyment spośród wszystkich recenzji, co pozwala na uzyskanie wartości ciągłej w przedziale $[0, 1]$. Ponieważ wyższa średnia wartość oznacza lepszy odbiór filmu przez krytyków, zmienna ta również jest stymulantą.
\end{itemize}

\section{Wstępna analiza danych}

\subsection{Statystyki opisowe}

W poniższej tabeli przedstawiono podstawowe statystyki opisowe dla wybranych zmiennych numerycznych: \texttt{audienceScore}, \texttt{tomatoMeter} oraz \texttt{runtimeMinutes}. Zawarte są w niej: wartość minimalna, maksymalna, średnia, mediana, odchylenie standardowe oraz współczynnik skośności.

\begin{table}[H]
\centering
\caption{Statystyki opisowe zmiennych numerycznych}
\begin{tabular}{lcccccc}
\toprule
\textbf{Zmienna} & \textbf{Min} & \textbf{Max} & \textbf{Średnia} & \textbf{Mediana} & \textbf{Odch. stand.} & \textbf{Skośność} \\
\midrule
audienceScore & 0 & 100 & 58.79 & 62.0 & 25.09 & -0.60 \\
tomatoMeter & 0 & 100 & 49.26 & 50.0 & 29.87 & -0.05 \\
runtimeMinutes & 41 & 319 & 106.68 & 103.0 & 22.53 & 1.34 \\
\bottomrule
\end{tabular}
\label{tab:stat_desc}
\end{table}

\subsection{Wizualizacja danych}

Aby lepiej zrozumieć rozkład zmiennych numerycznych oraz dominujące gatunki filmowe, przygotowano zestaw wykresów. Dla każdej zmiennej przedstawiono histogram znormalizowany do postaci procentowej. Na końcu pokazano udział najczęstszych gatunków filmowych.

\subsubsection*{Histogramy zmiennych numerycznych}

\begin{figure}[H]
    \centering
    \includegraphics[width=0.32\textwidth]{histogram_audienceScore_procentowo.png}
    \includegraphics[width=0.32\textwidth]{histogram_tomatoMeter_procentowo.png}
    \includegraphics[width=0.32\textwidth]{histogram_runtime_procentowo.png}
    \caption{Histogramy procentowe zmiennych numerycznych: \texttt{audienceScore}, \texttt{tomatoMeter}, \texttt{runtimeMinutes}}
    \label{fig:histograms_numeric}
\end{figure}

\subsubsection*{Występowanie gatunków filmowych}

\begin{figure}[H]
    \centering
    \includegraphics[width=0.8\textwidth]{rozkład_procentowy_gatunków.png}
    \caption{Najczęściej występujące gatunki filmowe (w \%)}
    \label{fig:genres}
\end{figure}


\subsection{Braki danych}

Wstępna analiza danych wykazała obecność braków w wielu kolumnach połączonego zbioru filmów i recenzji. W celu określenia zakresu problemu, przeprowadzono agregację liczby brakujących wartości w każdej kolumnie. Na tej podstawie podjęto decyzję o usunięciu kolumn, w których liczba brakujących danych była znaczna lub które nie wnosiły istotnej wartości do dalszej analizy. Wśród usuniętych znalazły się m.in. dane o dacie recenzji, tytule filmu, stanie recenzji, zawartości recenzji czy informacje o dystrybutorze.

Dla pozostałych kolumn, w których braki były marginalne i nie miały znaczącego wpływu na strukturę zbioru, wartości brakujące zostały zachowane. Końcowy zbiór danych użyty w analizie nie zawierał już żadnych braków danych, co uzyskano również dzięki przekształceniu i oczyszczeniu danych (np. parsowaniu oryginalnych ocen oraz mapowaniu zmiennych jakościowych na numeryczne).

\subsection{Obserwacje odstające}

W celu identyfikacji i eliminacji obserwacji odstających zastosowano klasyczną metodę IQR (ang. interquartile range). W szczególności skupiono się na czterech zmiennych liczbowych: \texttt{tomatoMeter}, \texttt{audienceScore}, \texttt{runtimeMinutes} oraz przekształconej zmiennej \texttt{originalScore}. Dla każdej z tych zmiennych obliczono wartości pierwszego i trzeciego kwartylu (Q1 i Q3), a następnie szerokość rozstępu międzykwartylowego (IQR = Q3 - Q1).

Obserwacje, które znajdowały się poniżej $Q1 - 1{,}5 \cdot IQR$ lub powyżej $Q3 + 1{,}5 \cdot IQR$, zostały uznane za odstające i usunięte ze zbioru. Takie podejście pozwoliło na zachowanie spójności danych i ograniczenie wpływu ekstremalnych wartości na późniejsze analizy, w szczególności w kontekście metod klasteryzacji i eksploracyjnych analiz statystycznych.



\section{Opis zastosowanych metod}

\subsection{Analiza skupień}

Analiza skupień (ang. \textit{Cluster Analysis}) to metoda eksploracyjna służąca do grupowania obserwacji w zbiory (klastry), tak aby elementy w obrębie jednego klastra były do siebie możliwie podobne, a między klastrami — jak najbardziej różne. Jest to metoda nienadzorowana, co oznacza, że nie wymaga wcześniejszego oznaczenia klas (etykiet) danych.

Miara podobieństwa pomiędzy obserwacjami może być różnie definiowana, najczęściej w oparciu o metryki odległości, takie jak odległość euklidesowa. Głównym celem analizy skupień jest minimalizacja wariancji wewnątrzklastrowej i maksymalizacja wariancji międzyklastrowej.

\subsection{Metody klasteryzacji}

\subsubsection{K-means}

Metoda \textit{K-means} polega na podziale zbioru danych na \( k \) rozłącznych klastrów poprzez iteracyjne przypisywanie punktów do najbliższych centroidów i aktualizowanie pozycji centroidów na podstawie średniej punktów w klastrze.

Funkcją celu w metodzie K-means jest minimalizacja sumy kwadratów odległości punktów od przypisanych centroidów:

\[
J = \sum_{i=1}^{k} \sum_{x_j \in C_i} \| x_j - \mu_i \|^2
\]

gdzie:
\begin{itemize}
    \item \( k \) — liczba klastrów,
    \item \( C_i \) — zbiór punktów należących do klastra \( i \),
    \item \( \mu_i \) — centroid klastra \( i \),
    \item \( x_j \) — punkt danych.
\end{itemize}

Metoda została pierwotnie zaproponowana przez MacQueena \cite{macqueen1967}.

\subsubsection{DBSCAN}

Metoda \textit{DBSCAN} (ang. Density-Based Spatial Clustering of Applications with Noise) identyfikuje klastry jako obszary o wysokim zagęszczeniu punktów. Kluczowe parametry to:
\begin{itemize}
    \item \( \varepsilon \) — promień sąsiedztwa (eps),
    \item \texttt{minPts} — minimalna liczba punktów w sąsiedztwie wymaganych do utworzenia klastra.
\end{itemize}

Punkty dzielone są na:
\begin{itemize}
    \item punkty rdzeniowe (core points),
    \item punkty graniczne (border points),
    \item szum (noise).
\end{itemize}

Dwa punkty \( x_i, x_j \) należą do tego samego klastra, jeśli istnieje ciąg punktów \( x_i = x_0, x_1, ..., x_n = x_j \), taki że każdy \( x_{l+1} \) znajduje się w sąsiedztwie \( \varepsilon \) punktu \( x_l \), który jest punktem rdzeniowym.

Metoda została zaproponowana przez Ester i in. \cite{ester1996}.

\subsubsection{Klasteryzacja hierarchiczna}

Klasteryzacja hierarchiczna buduje strukturę drzewiastą (dendrogram), w której punkty danych są łączone (lub dzielone) na podstawie podobieństwa. Stosowana jest najczęściej wersja aglomeracyjna, w której początkowo każdy punkt tworzy własny klaster, a następnie łączone są pary klastrów o najmniejszej odległości.

Typową miarą odległości między klastrami jest metoda średnich połączeń:

\[
D(A, B) = \frac{1}{|A| \cdot |B|} \sum_{x \in A} \sum_{y \in B} d(x, y)
\]

gdzie:
\begin{itemize}
    \item \( A, B \) — dwa klastry,
    \item \( d(x, y) \) — odległość pomiędzy punktami,
    \item \( |A|, |B| \) — liczność klastrów.
\end{itemize}

Metoda została szczegółowo opisana przez Murtagh i Contrerasa \cite{murtagh2012algorithms}.


\section{Rezultaty}

\subsection{Wyniki analizy}

W celu analizy struktury danych filmowych zastosowano trzy różne podejścia do klasteryzacji: metodę KMeans, DBSCAN oraz klasteryzację hierarchiczną. Każda z metod została przeprowadzona na odpowiednio przygotowanych i wystandaryzowanych zbiorach danych, a następnie poddana analizie wizualnej.

\subsubsection{KMeans}

Metoda KMeans została zastosowana na zestawie 9 zmiennych numerycznych, obejmujących m.in. oceny widzów i krytyków, czas trwania filmu oraz miary sentymentu. Liczba klastrów została dobrana na podstawie wykresu łokcia, który wskazał na optymalne rozbicie zbioru na $k = 3$ klastry. Analiza wykazała wyraźne oddzielenie grup filmów, m.in. takich z wysokimi ocenami krytyków, ale niskimi od widzów, oraz odwrotnie.

\begin{figure}[H]
\centering
\includegraphics[width=1\textwidth]{Metoda_łokcia_wybór_liczby_klastrów.png}
\caption{Wykres łokcia}
\end{figure}

Wyniki przedstawiono na wykresach 2-wymiarowych rzutów wybranych par zmiennych, pokolorowanych według przydziału do klastrów:

\begin{figure}[H]
\centering
\includegraphics[width=0.24\textwidth]{kmeans_audienceScore_vs_max_originalScore.png}
\includegraphics[width=0.24\textwidth]{kmeans_audienceScore_vs_mean_originalScore.png}
\includegraphics[width=0.24\textwidth]{kmeans_audienceScore_vs_median_originalScore.png}
\includegraphics[width=0.24\textwidth]{kmeans_audienceScore_vs_min_originalScore.png}
\includegraphics[width=0.24\textwidth]{kmeans_audienceScore_vs_runtimeMinutes.png}
\includegraphics[width=0.24\textwidth]{kmeans_audienceScore_vs_scoreSentiment.png}
\includegraphics[width=0.24\textwidth]{kmeans_audienceScore_vs_tomatoMeter.png}
\includegraphics[width=0.24\textwidth]{kmeans_audienceScore_vs_topCriticRatio.png}
\includegraphics[width=0.24\textwidth]{kmeans_max_originalScore_vs_mean_originalScore.png}
\includegraphics[width=0.24\textwidth]{kmeans_max_originalScore_vs_median_originalScore.png}
\includegraphics[width=0.24\textwidth]{kmeans_mean_originalScore_vs_median_originalScore.png}
\includegraphics[width=0.24\textwidth]{kmeans_min_originalScore_vs_max_originalScore.png}
\includegraphics[width=0.24\textwidth]{kmeans_min_originalScore_vs_mean_originalScore.png}
\includegraphics[width=0.24\textwidth]{kmeans_min_originalScore_vs_median_originalScore.png}
\includegraphics[width=0.24\textwidth]{kmeans_runtimeMinutes_vs_max_originalScore.png}
\includegraphics[width=0.24\textwidth]{kmeans_runtimeMinutes_vs_mean_originalScore.png}
\includegraphics[width=0.24\textwidth]{kmeans_runtimeMinutes_vs_median_originalScore.png}
\includegraphics[width=0.24\textwidth]{kmeans_runtimeMinutes_vs_min_originalScore.png}
\includegraphics[width=0.24\textwidth]{kmeans_runtimeMinutes_vs_scoreSentiment.png}
\includegraphics[width=0.24\textwidth]{kmeans_runtimeMinutes_vs_topCriticRatio.png}
\includegraphics[width=0.24\textwidth]{kmeans_scoreSentiment_vs_max_originalScore.png}
\includegraphics[width=0.24\textwidth]{kmeans_scoreSentiment_vs_mean_originalScore.png}
\includegraphics[width=0.24\textwidth]{kmeans_scoreSentiment_vs_median_originalScore.png}
\includegraphics[width=0.24\textwidth]{kmeans_scoreSentiment_vs_min_originalScore.png}
\includegraphics[width=0.24\textwidth]{kmeans_scoreSentiment_vs_topCriticRatio.png}
\includegraphics[width=0.24\textwidth]{kmeans_tomatoMeter_vs_max_originalScore.png}
\includegraphics[width=0.24\textwidth]{kmeans_tomatoMeter_vs_mean_originalScore.png}
\includegraphics[width=0.24\textwidth]{kmeans_tomatoMeter_vs_median_originalScore.png}
\includegraphics[width=0.24\textwidth]{kmeans_tomatoMeter_vs_min_originalScore.png}
\includegraphics[width=0.24\textwidth]{kmeans_tomatoMeter_vs_runtimeMinutes.png}
\includegraphics[width=0.24\textwidth]{kmeans_tomatoMeter_vs_scoreSentiment.png}
\includegraphics[width=0.24\textwidth]{kmeans_tomatoMeter_vs_topCriticRatio.png}
\includegraphics[width=0.24\textwidth]{kmeans_topCriticRatio_vs_max_originalScore.png}
\includegraphics[width=0.24\textwidth]{kmeans_topCriticRatio_vs_mean_originalScore.png}
\includegraphics[width=0.24\textwidth]{kmeans_topCriticRatio_vs_median_originalScore.png}
\includegraphics[width=0.24\textwidth]{kmeans_topCriticRatio_vs_min_originalScore.png}
\caption{Wizualizacje wyników klasteryzacji metodą KMeans}
\end{figure}

\subsubsection{DBSCAN}

Do metody DBSCAN wybrano trzy zmienne: \textit{audienceScore}, \textit{tomatoMeter} i \textit{runtimeMinutes}. Wartość parametru \textit{eps} została ustalona na podstawie wykresu k-Distance i wynosiła 0.35. DBSCAN poprawnie zidentyfikował kilka skupień, eliminując równocześnie obserwacje odstające jako szum (oznaczone klasą -1). 

Metoda ta wykazała się wysoką zdolnością separacji gęstych skupisk filmów o zbliżonych cechach, choć nieco mniej precyzyjną niż KMeans w przypadku mniej regularnych struktur.
\begin{figure}[H]
\centering
\includegraphics[width=1\textwidth]{k-Distance_Graph.png}
\caption{Wykres k-Distance}
\end{figure}

\begin{figure}[H]
\centering
\includegraphics[width=0.24\textwidth]{dbscan_audienceScore_vs_max_originalScore.png}
\includegraphics[width=0.24\textwidth]{dbscan_audienceScore_vs_mean_originalScore.png}
\includegraphics[width=0.24\textwidth]{dbscan_audienceScore_vs_median_originalScore.png}
\includegraphics[width=0.24\textwidth]{dbscan_audienceScore_vs_min_originalScore.png}
\includegraphics[width=0.24\textwidth]{dbscan_audienceScore_vs_runtimeMinutes.png}
\includegraphics[width=0.24\textwidth]{dbscan_audienceScore_vs_scoreSentiment.png}
\includegraphics[width=0.24\textwidth]{dbscan_audienceScore_vs_tomatoMeter.png}
\includegraphics[width=0.24\textwidth]{dbscan_audienceScore_vs_topCriticRatio.png}
\includegraphics[width=0.24\textwidth]{dbscan_max_originalScore_vs_mean_originalScore.png}
\includegraphics[width=0.24\textwidth]{dbscan_max_originalScore_vs_median_originalScore.png}
\includegraphics[width=0.24\textwidth]{dbscan_mean_originalScore_vs_median_originalScore.png}
\includegraphics[width=0.24\textwidth]{dbscan_min_originalScore_vs_max_originalScore.png}
\includegraphics[width=0.24\textwidth]{dbscan_min_originalScore_vs_mean_originalScore.png}
\includegraphics[width=0.24\textwidth]{dbscan_min_originalScore_vs_median_originalScore.png}
\includegraphics[width=0.24\textwidth]{dbscan_runtimeMinutes_vs_max_originalScore.png}
\includegraphics[width=0.24\textwidth]{dbscan_runtimeMinutes_vs_mean_originalScore.png}
\includegraphics[width=0.24\textwidth]{dbscan_runtimeMinutes_vs_median_originalScore.png}
\includegraphics[width=0.24\textwidth]{dbscan_runtimeMinutes_vs_min_originalScore.png}
\includegraphics[width=0.24\textwidth]{dbscan_runtimeMinutes_vs_scoreSentiment.png}
\includegraphics[width=0.24\textwidth]{dbscan_runtimeMinutes_vs_topCriticRatio.png}
\includegraphics[width=0.24\textwidth]{dbscan_scoreSentiment_vs_max_originalScore.png}
\includegraphics[width=0.24\textwidth]{dbscan_scoreSentiment_vs_mean_originalScore.png}
\includegraphics[width=0.24\textwidth]{dbscan_scoreSentiment_vs_median_originalScore.png}
\includegraphics[width=0.24\textwidth]{dbscan_scoreSentiment_vs_min_originalScore.png}
\includegraphics[width=0.24\textwidth]{dbscan_scoreSentiment_vs_topCriticRatio.png}
\includegraphics[width=0.24\textwidth]{dbscan_tomatoMeter_vs_max_originalScore.png}
\includegraphics[width=0.24\textwidth]{dbscan_tomatoMeter_vs_mean_originalScore.png}
\includegraphics[width=0.24\textwidth]{dbscan_tomatoMeter_vs_median_originalScore.png}
\includegraphics[width=0.24\textwidth]{dbscan_tomatoMeter_vs_min_originalScore.png}
\includegraphics[width=0.24\textwidth]{dbscan_tomatoMeter_vs_runtimeMinutes.png}
\includegraphics[width=0.24\textwidth]{dbscan_tomatoMeter_vs_scoreSentiment.png}
\includegraphics[width=0.24\textwidth]{dbscan_tomatoMeter_vs_topCriticRatio.png}
\includegraphics[width=0.24\textwidth]{dbscan_topCriticRatio_vs_max_originalScore.png}
\includegraphics[width=0.24\textwidth]{dbscan_topCriticRatio_vs_mean_originalScore.png}
\includegraphics[width=0.24\textwidth]{dbscan_topCriticRatio_vs_median_originalScore.png}
\includegraphics[width=0.24\textwidth]{dbscan_topCriticRatio_vs_min_originalScore.png}
\caption{Wizualizacje wyników klasteryzacji metodą DBSCAN}
\end{figure}

\subsubsection{Klasteryzacja hierarchiczna}

W przypadku klasteryzacji hierarchicznej zastosowano metody aglomeracyjne z wykorzystaniem metryki euklidesowej i metody \textit{ward}. Na podstawie dendrogramu wyodrębniono trzy główne klastry. Podobnie jak w przypadku KMeans, klastry dobrze różnicowały filmy pod kątem ocen i długości trwania.

\begin{figure}[H]
\centering
\includegraphics[width=1\textwidth]{dendrogram.png}
\caption{Dendrogram}
\end{figure}

\begin{figure}[H]
\centering
\includegraphics[width=0.24\textwidth]{hierarchical_audienceScore_vs_max_originalScore.png}
\includegraphics[width=0.24\textwidth]{hierarchical_audienceScore_vs_mean_originalScore.png}
\includegraphics[width=0.24\textwidth]{hierarchical_audienceScore_vs_median_originalScore.png}
\includegraphics[width=0.24\textwidth]{hierarchical_audienceScore_vs_min_originalScore.png}
\includegraphics[width=0.24\textwidth]{hierarchical_audienceScore_vs_runtimeMinutes.png}
\includegraphics[width=0.24\textwidth]{hierarchical_audienceScore_vs_scoreSentiment.png}
\includegraphics[width=0.24\textwidth]{hierarchical_audienceScore_vs_tomatoMeter.png}
\includegraphics[width=0.24\textwidth]{hierarchical_audienceScore_vs_topCriticRatio.png}
\includegraphics[width=0.24\textwidth]{hierarchical_max_originalScore_vs_mean_originalScore.png}
\includegraphics[width=0.24\textwidth]{hierarchical_max_originalScore_vs_median_originalScore.png}
\includegraphics[width=0.24\textwidth]{hierarchical_mean_originalScore_vs_median_originalScore.png}
\includegraphics[width=0.24\textwidth]{hierarchical_min_originalScore_vs_max_originalScore.png}
\includegraphics[width=0.24\textwidth]{hierarchical_min_originalScore_vs_mean_originalScore.png}
\includegraphics[width=0.24\textwidth]{hierarchical_min_originalScore_vs_median_originalScore.png}
\includegraphics[width=0.24\textwidth]{hierarchical_runtimeMinutes_vs_max_originalScore.png}
\includegraphics[width=0.24\textwidth]{hierarchical_runtimeMinutes_vs_mean_originalScore.png}
\includegraphics[width=0.24\textwidth]{hierarchical_runtimeMinutes_vs_median_originalScore.png}
\includegraphics[width=0.24\textwidth]{hierarchical_runtimeMinutes_vs_min_originalScore.png}
\includegraphics[width=0.24\textwidth]{hierarchical_runtimeMinutes_vs_scoreSentiment.png}
\includegraphics[width=0.24\textwidth]{hierarchical_runtimeMinutes_vs_topCriticRatio.png}
\includegraphics[width=0.24\textwidth]{hierarchical_scoreSentiment_vs_max_originalScore.png}
\includegraphics[width=0.24\textwidth]{hierarchical_scoreSentiment_vs_mean_originalScore.png}
\includegraphics[width=0.24\textwidth]{hierarchical_scoreSentiment_vs_median_originalScore.png}
\includegraphics[width=0.24\textwidth]{hierarchical_scoreSentiment_vs_min_originalScore.png}
\includegraphics[width=0.24\textwidth]{hierarchical_scoreSentiment_vs_topCriticRatio.png}
\includegraphics[width=0.24\textwidth]{hierarchical_tomatoMeter_vs_max_originalScore.png}
\includegraphics[width=0.24\textwidth]{hierarchical_tomatoMeter_vs_mean_originalScore.png}
\includegraphics[width=0.24\textwidth]{hierarchical_tomatoMeter_vs_median_originalScore.png}
\includegraphics[width=0.24\textwidth]{hierarchical_tomatoMeter_vs_min_originalScore.png}
\includegraphics[width=0.24\textwidth]{hierarchical_tomatoMeter_vs_runtimeMinutes.png}
\includegraphics[width=0.24\textwidth]{hierarchical_tomatoMeter_vs_scoreSentiment.png}
\includegraphics[width=0.24\textwidth]{hierarchical_tomatoMeter_vs_topCriticRatio.png}
\includegraphics[width=0.24\textwidth]{hierarchical_topCriticRatio_vs_max_originalScore.png}
\includegraphics[width=0.24\textwidth]{hierarchical_topCriticRatio_vs_mean_originalScore.png}
\includegraphics[width=0.24\textwidth]{hierarchical_topCriticRatio_vs_median_originalScore.png}
\includegraphics[width=0.24\textwidth]{hierarchical_topCriticRatio_vs_min_originalScore.png}
\caption{Wizualizacja wyników klasteryzacji hierarchicznej}
\end{figure}

\subsection{Porównanie metod}

Każda z zastosowanych metod klasteryzacji oferuje inne podejście do rozpoznawania struktury danych:

\begin{itemize}
    \item \textbf{KMeans} umożliwia wydajne i szybkie podziałanie danych na zadeklarowaną liczbę klastrów. Wymaga jednak uprzedniego określenia liczby grup oraz może nie radzić sobie dobrze w przypadku klastrów o nieregularnym kształcie lub zróżnicowanej gęstości.
    \item \textbf{DBSCAN} nie wymaga z góry ustalonej liczby klastrów, dobrze identyfikuje nieregularne struktury i potrafi wykryć szum. Jej skuteczność zależy jednak mocno od odpowiedniego doboru parametrów \textit{eps} i \textit{min\_samples}.
    \item \textbf{Klasteryzacja hierarchiczna} pozwala na wizualną analizę struktury danych poprzez dendrogram, co ułatwia decyzję o liczbie klastrów. Metoda ta jest jednak mniej skalowalna przy bardzo dużych zbiorach danych.
\end{itemize}

Pod względem jakości separacji klastrów, KMeans i hierarchiczna klasteryzacja dawały najbardziej czytelne i rozłączne grupy. DBSCAN natomiast pozwolił lepiej zidentyfikować filmy odstające, których nie dało się dobrze zaklasyfikować do typowych grup. Najwięcej wspólnych cech wykazywały wyniki KMeans i klasteryzacji hierarchicznej – w obu przypadkach pojawiły się podobne grupy filmów o skrajnych ocenach widzów i krytyków.


\section{Podsumowanie}
Projekt dostarczył wartościowych wniosków na temat różnic w ocenach filmów przez krytyków i widzów oraz wykazał przydatność metod klasteryzacji w analizie danych z Rotten Tomatoes. Wyniki mogą być pomocne w dalszych badaniach nad trendami w kinematografii i preferencjach widowni.

\subsection{Ocena realizacji celu}
Cel projektu został zrealizowany – udało się wyodrębnić grupy filmów o podobnych cechach i potwierdzić istnienie różnic w ocenach krytyków i widzów. Wyniki są zgodne z założeniami, choć w niektórych przypadkach (np. wpływ czasu trwania filmu na oceny) zależności były mniej wyraźne.

\subsection{Odniesienie do literatury}
Wyniki projektu są spójne z badaniami przytoczonymi w przeglądzie literatury:
\begin{itemize}
    \item Xu i in. (2022) również wykorzystali analizę skupień do badania recenzji filmowych, co potwierdza skuteczność tej metody w eksploracji danych.
    \item Verma i in. (2021) oraz Abimanyu i in. (2023) wykazali, że analiza sentymentu i ocen filmowych może skutecznie identyfikować trendy, co znalazło odzwierciedlenie w wynikach tego projektu.
    \item Nugraha i in. (2023) zastosowali zaawansowane metody klasyfikacji, co sugeruje możliwość dalszego rozwoju analizy w kierunku uczenia nadzorowanego.
\end{itemize}

\subsection{Ograniczenia i kierunki rozwoju}
\subsubsection{Ograniczenia}
Brak szczegółowej analizy wpływu gatunków filmowych na oceny (ze względu na ograniczoną reprezentację niektórych kategorii).
\subsubsection{Kierunki rozwoju}

\begin{itemize}
    \item Rozszerzenie analizy o klasyfikację nadzorowaną (np. przewidywanie ocen na podstawie recenzji tekstowych).
    \item Zastosowanie bardziej zaawansowanych metod NLP do analizy sentymentu.
    \item Uwzględnienie dodatkowych zmiennych, takich jak data premiery czy popularność aktorów.
\end{itemize}

\clearpage
\section{Bibliografia}
\begin{thebibliography}{9}

\bibitem{xu2022} 
Bin Xu, Cheng Chen, Jong-Hoon Yang, et al. (2022). Application of Cluster Analysis Technology in Visualization Research of Movie Review Data. \textit{Computational Intelligence and Neuroscience}. \url{https://doi.org/10.1155/2022/7756896}.

\bibitem{abimanyu2023} 
Abimanyu, A. J., Dwifebri, M., Astuti, H. (2023). Sentiment Analysis of Rotten Tomatoes Movie Reviews Using Logistic Regression and Information Gain Feature Selection. \textit{Journal of Physics: Conference Series}, 2630(1). \url{https://doi.org/10.1088/1742-6596/2630/1/012004}.

\bibitem{goutam2018}
Goutam, R., Garlapati, S., Chanda, B. (2018). Comparative Analysis of Movie Reviews from Critics and Audience Using Rotten Tomatoes Dataset. \textit{International Journal of Data Science and Analytics}, 5(3), 145-158. \url{https://doi.org/10.1007/s41060-017-0085-7}.

\bibitem{kaggle2025} Andrezaza. (2023). Clapper: Massive Rotten Tomatoes Movies and Reviews. \textit{Kaggle}. \url{https://www.kaggle.com/datasets/andrezaza/clapper-massive-rotten-tomatoes-movies-and-reviews}

\bibitem{macqueen1967} MacQueen, J. (1967). Some Methods for Classification and Analysis of Multivariate Observations. In: \textit{Proceedings of the Fifth Berkeley Symposium on Mathematical Statistics and Probability}, Vol. 1, pp. 281–297. University of California Press. \textit{Pośredni dostęp:} \url{https://projecteuclid.org/euclid.bsmsp/1200512992}

\bibitem{ester1996} Ester, M., Kriegel, H.-P., Sander, J., Xu, X. (1996). A Density-Based Algorithm for Discovering Clusters in Large Spatial Databases with Noise. In: \textit{Proceedings of the Second International Conference on Knowledge Discovery and Data Mining (KDD)}, pp. 226–231. \textit{Pośredni dostęp:} \url{https://www.aaai.org/Papers/KDD/1996/KDD96-037.pdf}

\bibitem{murtagh2012algorithms} Murtagh, F., Contreras, P. (2012). Algorithms for Hierarchical Clustering: An Overview. \textit{Wiley Interdisciplinary Reviews: Data Mining and Knowledge Discovery}, 2(1), 86--97. \url{https://doi.org/10.1002/widm.53}

\end{thebibliography}


\end{document}
